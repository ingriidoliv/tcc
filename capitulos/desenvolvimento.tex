\setlength{\absparsep}{18pt} % ajusta o espaçamento dos parágrafos do resumo
\begin{desenvolvimento}
 Historia IoT
 A IoT (Internet of Things, Internet das Coisas), começou bem antes da internet, sua raiz vem da tecnologia RFID – Radio Frenquency Identification, essa tecnologia resulta em identificação por rádio frequência, a primeira vez que foi usada foi na 2ª guerra mundial, para identificar se o avião era amigo ou inimigo. O avião ao captar o sinal do radar, deveria refletir o sinal com suas características, que seria sistema passivo ou sistema ativo e assim iria permitir ao radar identificar qual avião seria do mesmo grupo ou do grupo inimigo. Hoje essa tecnologia vem ganhando espaço no mercado em relação a lojas de roupa, e supermercados e comércios em geral, assim fazendo com que lojas tenham mais segurança em casos de roubos, como por exemplo, caso a etiqueta não tenha sido identificada no caixa, ao cliente passar pela porta da loja o alarme é disparado pelo sistema antirroubo. (MINERVA; BIRU; ROTONDI, 2015).
 Em 1999 no laboratório do MIT, Neil Gershendfeltdt lança o livro “When Things Starts To Think”, em que ele descreve “as coisas começam a usar a Net”. Em 2002 na revista Forbes Magazine, o pesquisador do Auto-ID Center, Kevin Ashton, usa a expressão “Internet of Things” pela primeira vez. (FACCIONI FILHO, 2016b).
 Em 2008 acontece a primeira conferencia internacional sobre o tema internet das coisas na Suiça. Nessa conferencia são discutidos todos os assuntos relacionados em internet das coisas, temas como RFID, sensoriamento, aspectos de negócios e tecnologias de conexão e conversão de protocolos, assim dando origem a um amplo campo de debates e evoluções técnicas, assim consolidando o cenário de internet das coisas.
 A internet das coisas é um conceito que está fora do âmbito das tecnologias, pois não deriva delas, e sim as utiliza para cumprir series de funcionalidades em objetos, fazendo com que conecte objetos a internet. Deixando tudo automatizado, facilitando o uso de vários equipamentos que podemos colocar na internet, assim podemos dizer que a IoT é a tecnologia que irá se desenvolver cada vez mais, para que tudo possa ser mais funcional e que “tudo” esteja conectado a internet, pois é isso que ela faz, conecta todos os objetos dos mais simples, aos mais complexos a internet.
 Evolução
  
\end{desenvolvimento}